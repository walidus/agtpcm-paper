% This is samplepaper.tex, a sample chapter demonstrating the
% LLNCS macro package for Springer Computer Science proceedings;
% Version 2.20 of 2017/10/04
%
\documentclass[runningheads]{llncs}
%
\usepackage{graphicx}
% Used for displaying a sample figure. If possible, figure files should
% be included in EPS format.
%
% If you use the hyperref package, please uncomment the following line
% to display URLs in blue roman font according to Springer's eBook style:
% \renewcommand\UrlFont{\color{blue}\rmfamily}

\begin{document}
%
\title{Automatic Generation and Translation of Process Collaboration Models}
%
%\titlerunning{Abbreviated paper title}
% If the paper title is too long for the running head, you can set
% an abbreviated paper title here
%
\author{Frederik Bischoff \and Walid Fdhila}

%\author{Frederik Bischoff\inst{1} \and Walid Fdhila\inst{1}}
%
\authorrunning{F. Author et al.}
% First names are abbreviated in the running head.
% If there are more than two authors, 'et al.' is used.
%
\institute{Faculty of Computer Science, University of Vienna, Austria}
%
\maketitle              % typeset the header of the contribution
%
\begin{abstract}
In the research field of business process models and techniques, researchers can only rely on a repository of centralized, intra-organizational processes to use as support of their work. But regarding decentralized, cross-organizational models, they face the problem that there is a lack of available model examples. Within this work this lack is tried to be tackled. Thereby, a concept is introduced how to generate collaborative business processes randomly by following a \textit{top-down approach}, which first generates the choreography model and then derives the public and private models of each process participant from it. Additionally, a possibility for specifying and imposing global compliance rules onto the collaboration is elaborated. The conception of this random process generator is prototypically implemented within an existing framework in order to evaluate the process.

\keywords{Process Collaboration  \and Automatic Generation \and Process Models \and Compliance Rules.}
\end{abstract}
%
%
%
\section{Introduction}

%
% ---- Bibliography ----
%
% BibTeX users should specify bibliography style 'splncs04'.
% References will then be sorted and formatted in the correct style.
%
\bibliographystyle{splncs04}
\bibliography{bibliography}
%
\end{document}
