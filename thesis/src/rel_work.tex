This work focuses on the generation of collaborative business process models. Thus, the thereby introduced method of generating processes randomly can be compared with work that also generate business process models but differ from the input source of generation and it's algorithm. In comparison to the purpose of this work, which is building a repository of process collaborations to facilitate further research, they mainly have the goal to simplify the process of business process modeling by reducing the time for process acquisition and also generation. At the end, a similar research is introduced which also follows an approach of random process generation with the goal to facilitate further research with process examples.\\
There is research on generating processes from natural language \cite{natlang1}\cite{natlang2}. In \cite{natlang1}, Friedrich et al. introduced an approach to generate BPMN process models from natural language texts by utilizing syntax parsing and semantic analyzing mechanism in combination with anaphora resolution\footnote{Resolving what a pronoun or noun refers to.}. The result of the parsing algorithm is a declarative model that includes the extracted actions, actors and their dependencies. This model then serves as basis for the BPMN model composition. Compared to that, Krzysztof et al. introduced an transformation approach for structured natural language in form of SBVR\footnote{Semantics of Business Vocabulary and Business Rules} in \cite{natlang2}.\\
There are also approaches which generate BPMN models on the basis of UML\footnote{Unified Modeling Language} use cases \cite{relUML1} \cite{relUML2} or sequence diagrams \cite{relUML3}.\\
Another interesting, recent research focuses on the generation of process models based on constraint programming \cite{relCP}. Therein, Wisniewski et al. introduced an approach which uses semi-structured information about process activities along with their execution conditions as input for a constraint satisfaction problem (CSP). On the basis of this CSP, a constraint solver generates synthetic execution logs of all valid execution sequences. At last, the BPMN model is generated with the aid of process mining techniques. Their approach is similar to the one introduced in this work. The conceived model generation algorithm in this work also takes user defined dependencies between tasks (see Chapter \ref{sec:conception_compliance}) into account.\\
In \cite{relPLG}, Burattin also introduced a tool for generating BPMN process models randomly. In contrast to this work, which focuses on process collaborations, the tool only supports basic process models with the components task, exclusive and parallel gateways. The tool also supports user-defined parameters to influence the model outcome in terms of number of node types and level of branching. It is also possible to import existing models in order to evolve them. But in contrary to this work, it's main purpose is less the generation of random process models than the combination with the possibility to simulate the processes in order to obtain execution logs for testing process mining algorithms.