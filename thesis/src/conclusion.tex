\section{Conclusion}
In this work, the lack of available collaborative process models represented in BPMN/XML for different research purposes, such as process mining or change propagation, was addressed. The main contribution within this work is the creation of a model repository by implementing a process generator which generates decentralized, cross-organizational models based on several, user-specified parameters. Thereby, an introduction into the two possible approaches of how to build collaborative processes, with all the models representing the different process perspectives, has been provided. It was argued that the followed \textit{top-down approach}, if implemented correctly, already provides good preconditions for ensuring compatibility between the public models as well as consistency between the public and the private model of one partner by deriving the public from the generated choreography model and enhancing the public model with private tasks to obtain the private model. Additionally to the model-specific parameters, also the possibility of specifying and imposing compliance rules was examined and evaluated by implementing a prototypical compliance rule component which supports a small subset of established compliance patterns in order to also examine the semantical correctness of the models. The main focus of this work was on the conception of the automatic generator and the therefor necessary components and algorithms. The benefit of expressing the models as a RPST was argued as well as the importance of tracking the model at any point during the generation process in order to comply with the user-specified parameters, which serve as boundaries for the random generation. Therefore, the \textit{Model Tracking} component and the concept of already reserved but not yet consumed interactions was explained in detail. Due to the simplicity of the underlying algorithms, the deriving and enhancement of the public and private model, as well as the translation of the internal model representation to BPMN/XML was merely discussed superficially by explaining the node mapping between the different models as well as the mapping to the BPMN/XML elements.\\
Finally, a performance analysis was performed, which showed the influence of different build parameters on the performance of the generation process as well as on the result. Thereby, it was shown that the number of specified nodes influences the performance of the generation process linearly. But the analysis also exposed room for improvement of the generation process in general and especially regarding the specification and imposition of compliance rules, which will be addressed in the next chapter.


\section{Future Work}
In the context of the conducted performance analysis, it was exposed that for generation processes with imposed compliance rules, combined with an amount of exclusive gateways greater than five, the outcome was not satisfying. By generating and imposing random compliance rules at a ratio of min. 1 : 5 to the amount of interactions, the build was only successful at approx. 1 out of 4 times. This was not unexpected and is a result of the characteristics of the supported compliance patterns, which are more difficult to assign the more they are interdependent. This problem could be solved by supporting more compliance patterns, especially \textit{P Exclusive Q}, which requires Interaction P to be on a different exclusive path than Interaction Q \cite{compliance_patterns}. In addition to that, the overall performance of assigning the compliance rules onto the model, could be improved by introducing a caching concept for possible positions instead of looping through the whole model for each compliance rule, which can be highly time consuming if the model is large. It would also be interesting for this problem to utilize a constraint solving framework like \textit{CHOCO}\footnote{An Open-Source java library for constraint programming - http://http://www.choco-solver.org/}. For the general build process, it could be conceivable to implement more parameters that allow the user to influence the model outcome even more. Thereby, for example, a parameter which influences the branch selection dependent on the degree of nested branching by dynamically favoring branches which are already highly nested or those that are not, depending on the parameter setting. 
\clearpage