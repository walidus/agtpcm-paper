\section{Abstract}

In the research field of business process models and techniques, researchers can only rely on a repository of centralized, intra-organizational processes to use as support of their work. But regarding decentralized, cross-organizational models, they face the problem that there is a lack of available model examples. Within this work this lack is tried to be tackled. Thereby, a concept is introduced how to generate collaborative business processes randomly by following a \textit{top-down approach}, which first generates the choreography model and then derives the public and private models of each process participant from it. Additionally, a possibility for specifying and imposing global compliance rules onto the collaboration is elaborated. The conception of this random process generator is prototypically implemented within an existing framework in order to evaluate the process.

\section{Zusammenfassung}

Zur Unterstützung bei der Forschung an Geschäftsprozessmodellen und -techniken stehen lediglich ausreichend zentrale, unternehmensinterne Beispielprozesse zur Verfügung. Bei dezentralen, organisationsübergreifenden Prozessmodellen mangelt es allerdings an ausreichend verfügbaren Modellbeispielen. Im Rahmen dieser Arbeit wird versucht diesen Mangel zu beheben. Es wird ein Konzept vorgestellt, wie kollaborative Geschäftsprozesse randomisiert generiert werden können. Dabei wird ein "Top-Down" Ansatz verfolgt, der zunächst das Choreographie Modell generiert und anschließend die öffentlichen und privaten Modelle jedes Prozessteilnehmers daraus ableitet. Darüber hinaus wurde eine Möglichkeit erarbeitet, globale Compliance-Regeln zu definieren und dem Generierungsprozess als Rahmen aufzuerlegen. Um den erarbeiteten Prozess evaluieren zu können, wurde innerhalb eines bestehenden Frameworks ein Prototyp implementiert.
